\paragraph{1}
Dato un grfo connesso \textit{G}, dimostrare che - essendo connesso - ci sono almeno due nodi con lo stesso grado: $ \exists u,v \in V : deg(u) = deg(v) $ \\
\textbf{Dimostrazione}: per assurdo consideriamo un grafo connesso che però abbia tutti i nodi di grado diverso. Ci troveremo necessariamente ad avere un nodo con grado $n-1$, un altro con grado $n-2$, fino ad uno con grado $n-n = 0$. Un grado però non può avere grado \textit{0}, a meno che non si tratti di un grafo sconnesso.

\paragraph{2}
Dato un grafo \textit{G}, con \textit{V} nodi e \textit{E} vertici, in cui ciascun nodo abbia grado $\geq 2$, dimostrare che il grafo \textit{G} è ciclico. \\
\textbf{Dimostrazione}: ???

\paragraph{3}
Dato un albero \textit{T} con $n \geq 2$ nodi, dimostrare che esistono almeno 2 foglie. \\
\textbf{Dimostrazione}: visitiamo il grafo partendo da un nodo arbitrario tramite la procedura \textit{DFS}. Questa troverà necessariamente un nodo il cui grado è più alto di tutti gli altri e fermerà la ricorsione. Si tratta della prima foglia. \\
Applicando la stessa procedura, a partire dalla foglia appena individuata, sicuramente, ancora una volta, l'algoritmo, applicherà la ricorsione fino a trovare un nodo il cui grado sia diverso da quello della foglia usata in questo caso come radice. Si tratta della seconda foglia.

\paragraph{4}
Dato un grafo diretto \textit{G}, si vuole calcolare il tempo per calcolarne il grado uscente e normale di ciascun vertice, usando liste di adiacenza. Successivamente, utilizzando matrici. \\
\textbf{Soluzione}: ???
\newpage

\paragraph{5}
Dato un vettore di padri, generato da una visita tramite \textit{BFS}, calcolare il vettore delle distanze in $O(n)$, senza che queste siano salvate tramite la visita. \\
\textbf{Soluzione parziale}: \hfill
\begin{algorithm}
	\caption{Esercizio 5}\label{alg:es5}
	\begin{algorithmic}[1]
		\Function{contaDistanze}{}
		\For { $i \gets 1$ to \textit{n}}
			\State $d \gets 0$
			\State $j \gets i$
			\While { $P(j) \neq j$ }
				\State $j \gets P(j)$
				\State $i++$
			\EndWhile
			\State $Dist(i) \gets d$
		\EndFor
		\EndFunction
	\end{algorithmic}
\end{algorithm}
\hfill
In questo caso, il \textit{while} ci porterebbe a calcolare troppe volte le distanze, anche quando non necessario. Infatti, se dopo un nodo \textit{x}, occorre calcolare la distanza del suo figlio \textit{x+1}, sicuramente questo avrà distanza $Dist(x)+1$, senza dover ricalcolarla da zero. \\
Per ovviare a questo problema e raggiungere un costo di $O(n)$:
\begin{algorithm}
	\caption{Esercizio 5}\label{alg:es5-2}
	\begin{algorithmic}[1]
		\Function{contaDistanze}{$i$}
		\State $Dist(i) \gets -1$
		\For { $i \gets 1$ to \textit{n}}
			\If { $Dist(i) = -1$ }
				\State $calcola(i)$
			\EndIf
		\EndFor
		\EndFunction
	\end{algorithmic}
	\begin{algorithmic}[1]
		\Function{calcola}{$i$}
		\If { $i = u$ }
			\State $ Dist(i) = 0 $
		\ElsIf { $Dist(P[i]) \neq -1$ }
			\State $ Dist(u) = Dist(P[i]) + 1$
		\Else
			\State $int d = calcola(P[i])$
			\State $Dist(i) = d+1$
		\EndIf
		\EndFunction
	\end{algorithmic}
\end{algorithm}
\newpage

\paragraph{6}
Ho un grafo connesso, i cui archi sono colorati. Voglio dimostrare la validità di un algoritmo che produca un \textit{albero di copertura} che utilizza il minor numero di colori distinti. Algoritmo proposto: \hfill
\begin{algorithm}
	\caption{Esercizio 6}\label{alg:es6}
	\begin{algorithmic}[1]
		\Function{algo}{}
		\State $Sol \gets \phi$
		\While {$ E \neq \phi $}
		\If { esiste un arco in \textit{E} con un colore già inserito in \textit{Sol} }
		\State estrai quell'arco
		\Else
		\State estrai arco \textit{e} che ha colore che occorre più spesso in \textit{E}
		\EndIf
		\If { \textit{e} non crea ciclo in \textit{Sol} }
		\State $Sol \gets Sol \cup \{e\}$
		\EndIf
		\EndWhile
		\State \textbf{return} \textit{Sol}
		\EndFunction
	\end{algorithmic}
\end{algorithm} \\ \hfill
\textbf{Soluzione:} Questo algoritmo fa leva sui colori che compaiono maggiormente nel grafo. Il problema è che esistono altri criteri che valorizzano dei colori piuttosto che altri, che vanno a scontrarsi con la regola del \textit{colore che compare più volte}: per esempio, che un dato arco di un dato colore è l'unico che rende il grafo connesso.

\paragraph{7}
Ho \textit{n} numeri ordinati in un vettore, contenente soltanto valori \textit{1} e \textit{2}: scrivere un algoritmo che in tempo $O(\log{n})$ restituisca il numero di \textit{1} contenuti nel vettore. \\
\textbf{Suggerimento}: usare la ricerca binaria.
\begin{algorithm}
	\caption{Esercizio 7}\label{alg:es7}
	\begin{algorithmic}[1]
		\Function{algo}{}
		\If { $V[n] = 1$ }
		\Return n
		\EndIf
		\If { $V[n] = 2$ }
		\Return $\phi$
		\EndIf
		\Return RicercaBinaria(1, n-1)
		\EndFunction
	\end{algorithmic}
	\begin{algorithmic}[1]
		\Function{RicercaBinaria(i, j)}{}
		\State $m \gets \lfloor\frac{i+j}{2}\rfloor$
		\If $V[m] = 1$ \&\& $V[m+1] = 2$
		\Return m
		\EndIf
		\If $V[m] = 1$
		\Return RicercaBinaria(m+1, j)
		\EndIf
		\Return RicercaBinaria(i, m-1)
		\EndFunction
	\end{algorithmic}
\end{algorithm} \\ \hfill
\newpage

\paragraph{8}
Ho \textit{n} numeri ordinati in un vettore, contenente soltanto valori \textit{1}, \textit{2} e \textit{3}: scrivere un algoritmo che in tempo $O(\log{n})$ restituisca il numero di \textit{1} e \textit{3}. \\
\textbf{Suggerimento}: usare la ricerca binaria.
\begin{algorithm}
	\caption{Esercizio 7}\label{alg:es8}
	\begin{algorithmic}[1]
		\Function{algo}{}
		\If { $V[n] = 1$ }
		\Return n
		\EndIf
		\If { $V[n] = 3$ }
		\Return n
		\EndIf
		\Return RicercaBinaria(1, n-1)
		\EndFunction
	\end{algorithmic}
	\begin{algorithmic}[1]
		\Function{RicercaBinaria(i, j)}{}
		\State \ldots
		\EndFunction
	\end{algorithmic}
\end{algorithm} \\ \hfill

\paragraph{9}
Dato un grafo diretto, occorre dimostrare se può essere stato generato tramite una visita \textit{DFS} utilizzando liste di adiacenza. Se sì, dimostrare come è strutturato il grafo. \\
\textbf{Soluzione}: partendo dal grafo proposto, scrivere le liste di adiacenza. Una volta scritte, partendo da quelle - e nello specifico dal nodo che ha solo archi (di visita) entranti -, verificare che con queste posso arrivare - tramite la visita - a ottenere lo stesso grafo/albero.
\newpage

\paragraph{Es1 (4pti)}
Si consideri il grafo G in figura e l'albero T degli archi marcati. Partendo da quali nodi l'albero T può essere stato prodotto da una visita DFS e/o BFS? Motivare bene la risposta. \\
\textbf{Soluzione}:
\begin{itemize}
	\item \textit{BFS}: con BFS necessariamente non si può essere partiti da tutti i nodi esclusi 8 e 5, perché avrebbe dovuto visitare prima tutti gli adiacenti, e così non è per tutti gli altri nodi. Partendo da 5, è corretta l'iterazione fino a raggiungere il nodo 2, per il quale, però, si sarebbe dovuto visitare non solo il 6, ma anche il 2, prima di poter continuare con l'iterazione sugli adiacenti: anche il 5 è scarato. Analogamente, partendo da 8, è corretta l'iterazione, dando per vero che il 2 è stato il primo visitato, l'iterazione è corretta fino a 3, dove ancora una volta, va in errore, perché avrebbe dovuto visitare prima il 4 partendo dallo stesso 3, invece di passare per il 5. Di conseguenza, non è possibile la BFS.
	\item \textit{DFS}: in un albero derivato dalla visita DFS deve essere escluso necessariamente un qualsiasi arco cross. Partendo da 8, seguendo con 9, la DFS selezionerebbe necessariamente l'arco {9,1}, seguendo con 1, o l'arco {1,2} o {1,9}, seguendo con 2, l'arco {2,1}, seuendo con 7, l'arco {7,6}. Tutti gli archi non inseriti dalla DFS devono essere necessariamente archi all'indietro. Di conseguenza, poiché - da qualsiasi nodo si parta - si incontrerà un arco di cross, l'albero in figura non può essere generato da una DFS. Quindi per dimostrare che - a partire da un albero - questo non può essere stato generato da una DFS, occorrerà trovare, tra gli archi non inclusi nell'albero, un arco che connette due nodi che, nell'albero, non sono uno l'antenato dell'altro (ovvero, i cross).
\end{itemize}
\newpage

\paragraph{Es2 (8pti)}
Elaborare lo pseudocodice di un algoritmo che preso in input un grafo non diretto e connesso G su un nodo u, il vettore dei padri P prodotto dalla BFS da u in G e l'arco {v,w} di G ritorna true se e solo se la rimozione di {v,w} non cambia la distanza da u. Il tutto in O(n). \\
\textbf{Soluzione}:
\begin{algorithm}
	\caption{Esercizio 2}\label{alg:es2}
	\begin{algorithmic}[1]
		\Function{algo}{$P, \{v,w\}$}
			\State costruisci l'albero T a partire dal vettore dei padri P, in $O(n)$
			\State costruisci il vettore delle distanza D con BFS a partire dall'albero T, in $O(n)$
			\If { $P[v] \neq w$ and $P[w] \neq v$ }
				\State \Return true
			\Else
				\If { $P[v] = w$ }
					\For {$k \in adj(v)$}
						\If { $D[k] = D[v]$ and $k \neq w$ }
							\State \Return true
						\EndIf
					\EndFor
				\ElsIf { $P[w] = v$ }
					\For {$k \in adj(w)$}
						\If { $D[k] = D[w]$ and $k \neq v$ }
							\State \Return true
						\EndIf
					\EndFor
				\EndIf
			\EndIf
			\State \Return false
		\EndFunction
	\end{algorithmic}
\end{algorithm} \\ \hfill
La costruzione dell'albero con il vettore dei padri P costa $O(n)$, sommata alla costruzione del vettore delle distane D con BFS sull'albero T che costa $O(n)$, arriviamo a $2 \times O(n)$. Infine, nella sezione finale ha costo rilevando soltanto il check sugli adiacenti di $w$ o $v$, che può costare al massimo $n-1$: $O(n)$. Quindi $2 	\times O(n) + O(n)$ equivale asintoticamente a $O(n)$.
\newpage

\paragraph{Es3 (8pti)}
Per un grafo diretto G diciamo che un arco {u,v} è interno se u e v appartengono alla stessa componente fortemente connessa, e altrimenti diciamo che esterno. In relazione con una qualsiasi DFS, rispondere alle seguenti domande:
\begin{enumerate}
	\item un arco all'indietro può essere esterno?
	\item un arco in avanti può essere interno?
	\item un arco di attraversamento può essere interno?
	\item un arco di attraversamento deve essere esterno?
\end{enumerate}
Per ognuna, in caso affermativo, esibire un esempio, altrimenti dimostrare l'impossibilità. \\
\textbf{Soluzione:}
\begin{enumerate}
	\item un arco all'indietro è necessariamente interno perché sarà - in un albero - la componente fondamentale per determinare una componente fortemente connessa, perché determina un ciclo con la concatenazione di nodi, da ascendente a discendente (vedi algoritmo di Tarjan, con definizione di c-root). $\Rightarrow$  deve essere interno;
	\item un arco in avanti può essere interno considerando il caso in cui la componente considerata coinvolga un arco all'indietro che raggiunga o il nodo d'origine dell'arco in avanti, o vada più in alto, e che parta da un nodo che è il nodo di origine dell'arco in avanti, o più in basso. $\Rightarrow$ può essere interno;
	\item un arco di attraversamento può essere interno anche semplicemente considerando il caso in cui abbiamo un albero diretto di tre nodi. Sostanzialmente, se i due nodi figli sono legati da un arco di attraversamento, può esistere il caso in cui a partire da quello raggiungo vi sia un arco all'indietro verso la radice. $\Rightarrow$ può essere interno;
	\item la risposta fornita alla domanda precedente stabilisce un controesempio alla seguente domanda. Pertanto, no: un arco di attraversamento non \textit{deve} necessariamente interno.
\end{enumerate}
\newpage

\paragraph{Es4 (10pti)}
In un museo c'è un lungo corridoio rettilineo in cui sono esposti $n$ quadri nelle posizioni $q_{1} < q_{2} < \ldots < q_{n}$. Un custode può sorvegliare i quadri che si trovano al più a distanza 5 dalla sua posizione. Vogliamo sorvegliare tutti i quadri con un numero minimo di custodi. Dare lo pseudocodice di un algoritmo che trova le posizioni di un numero minimo di custodi che sorvegliano gli $n$ quadri. La complessità dell'algoritmo deve essere $O(n)$. Dare la dimostrazione di correttezza dell'algoritmo proposto. \\
\textbf{Soluzione}:
\begin{algorithm}
	\caption{Esercizio 4}\label{alg:es4}
	\begin{algorithmic}[1]
		\Function{algo}{$n$}
			\State inizializzo C come vettore di soli 0
			\For { $i \gets 1$ to $n$ }
				\If { $i\%11 = 6$ }
					\State $C[i] \gets 1$
				\EndIf
			\EndFor
			\If { $n\%11 < 6$ }
				\State $C[n] = 1$
			\EndIf
			\State \Return C
		\EndFunction
	\end{algorithmic}
\end{algorithm} \\ \hfill

\paragraph{Es1 SIMULAZIONE}
Abbiamo una matrica $n\times n$: verificare se si può raggiungere la cella in basso a destra partendo dalla cella in alto a sinistra, senza toccare le celle che contengono $1$ e muovendosi solo in basso o a destra. \\
Soluzione parziale:
\begin{algorithm}
	\caption{Es1}\label{alg:es1-sim}
	\begin{algorithmic}[1]
		\Function{es1}{$i,j$}
			\If { $M[i,j] = 1$ }
				\State \Return false
			\EndIf
			\If { $i = 1$ and $j = 1$ }
				\State \Return true
			\EndIf
			\If { $i = 1$ }
				\State \Return $es1(i,j-1)$
			\EndIf
			\If { $j = 1$ }
				\State \Return $es1(i-1,j)$
			\EndIf
			\State \Return $es1(i-1,j)$ or $es1(i,j-1)$
		\EndFunction
	\end{algorithmic}
\end{algorithm} \\ \hfill
Adesso basta invocare questa procedura per $n,n$, per trovare la rispota al problema. \\
Questo algoritmo è poco efficiente, infatti ricalcolo più volte gli stessi sottoproblemi (simile al calcolo di \textit{Fibonacci} per ricorsione). \\
Infatti se stendiamo l'albero di ricorsione abbiamo un albero binario di altezza $2n$, raggiungendo una complessità di $O(4^n)$. \\
Possiamo migliorare la complessità tramite la tecnica della programmazione dinamica. Aggiungiamo una tabella in cui salviamo i risultati dei sottoproblemi, e una condizione che, invece di ricalcolare il sottoproblema, prima controlla nella tabella se già è stato risolto:
\begin{algorithm}
	\caption{Es1}\label{alg:es1-sim-2}
	\begin{algorithmic}[1]
		\Function{es1}{$i,j$}
			\If { $T[i,j] \neq -1$ }
				\State \Return $T[i,j]$
			\EndIf
			\If { $M[i,j] = 1$ }
				\State $T[i,j]$ = false
				\State \Return false
			\EndIf
			\If { $i = 1$ and $j = 1$ }
				\State $T[i,j]$ = true
				\State \Return true
			\EndIf
			\If { $i = 1$ }
				\State \Return $es1(i,j-1)$
			\EndIf
			\If { $j = 1$ }
				\State \Return $es1(i-1,j)$
			\EndIf
			\State \Return $es1(i-1,j)$ or $es1(i,j-1)$
		\EndFunction
	\end{algorithmic}
\end{algorithm} \\ \hfill
Questa seconda versione è più efficiente, ma rimane il problema che per calcolare la nostra risposta le calcoliamo tutte. \\
Se invece di usare la ricorsione calcoliamo noi il percorso partendo direttamente dalla cella $n,n$ possiamo migliorare ulteriormente la complessità.
\begin{algorithm}
	\caption{Es1}\label{alg:es1-sim-3}
	\begin{algorithmic}[1]
		\Function{es1}{$i,j$}
			\For { $i\gets 1$ to $n$ }
				\For { $j\gets 1$ to $n$ } 
					\If { $M[i,j] = 1$ }
						\State $T[i,j]$ = false
					\ElsIf { $i = 1$ and $j = 1$ }
						\State $T[i,j]$ = true
					\State \Return true
					\ElsIf { $i = 1$ }
						\State $T[i,j] = T[i,j-1]$
					\ElsIf { $j = 1$ }
						\State $T[i,j] = T[i-1,j]$
					\Else
						\State $T[i,j] = T[i-1,j]$ or $T[i,j-1]$
					\EndIf
				\EndFor
			\EndFor
			\State \Return $T[n,n]$
		\EndFunction
	\end{algorithmic}
\end{algorithm} \\ \hfill

Questa procedura ha complessità $n^2$. \\
Una volta calcolata la tabella, se vogliamo trovare il cammino, basta risalire la tabella al contrario. \\