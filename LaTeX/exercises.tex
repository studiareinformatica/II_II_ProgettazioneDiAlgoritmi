\paragraph{1}
Dato un grfo connesso \textit{G}, dimostrare che - essendo connesso - ci sono almeno due nodi con lo stesso grado: $ \exists u,v \in V : deg(u) = deg(v) $ \\
\textbf{Dimostrazione}: per assurdo consideriamo un grafo connesso che però abbia tutti i nodi di grado diverso. Ci troveremo necessariamente ad avere un nodo con grado $n-1$, un altro con grado $n-2$, fino ad uno con grado $n-n = 0$. Un grado però non può avere grado \textit{0}, a meno che non si tratti di un grafo sconnesso.

\paragraph{2}
Dato un grafo \textit{G}, con \textit{V} nodi e \textit{E} vertici, in cui ciascun nodo abbia grado $\geq 2$, dimostrare che il grafo \textit{G} è ciclico. \\
\textbf{Dimostrazione}: ???

\paragraph{3}
Dato un albero \textit{T} con $n \geq 2$ nodi, dimostrare che esistono almeno 2 foglie. \\
\textbf{Dimostrazione}: visitiamo il grafo partendo da un nodo arbitrario tramite la procedura \textit{DFS}. Questa troverà necessariamente un nodo il cui grado è più alto di tutti gli altri e fermerà la ricorsione. Si tratta della prima foglia. \\
Applicando la stessa procedura, a partire dalla foglia appena individuata, sicuramente, ancora una volta, l'algoritmo, applicherà la ricorsione fino a trovare un nodo il cui grado sia diverso da quello della foglia usata in questo caso come radice. Si tratta della seconda foglia.

\paragraph{4}
Dato un grafo diretto \textit{G}, si vuole calcolare il tempo per calcolarne il grado uscente e normale di ciascun vertice, usando liste di adiacenza. Successivamente, utilizzando matrici. \\
\textbf{Soluzione}: ???
\newpage

\paragraph{5}
Dato un vettore di padri, generato da una visita tramite \textit{BFS}, calcolare il vettore delle distanze in $O(n)$, senza che queste siano salvate tramite la visita. \\
\textbf{Soluzione parziale}: \hfill
\begin{algorithm}
	\caption{Esercizio 5}\label{alg:es5}
	\begin{algorithmic}[1]
		\Function{contaDistanze}{}
		\For { $i \gets 1$ to \textit{n}}
			\State $d \gets 0$
			\State $j \gets i$
			\While { $P(j) \neq j$ }
				\State $j \gets P(j)$
				\State $i++$
			\EndWhile
			\State $Dist(i) \gets d$
		\EndFor
		\EndFunction
	\end{algorithmic}
\end{algorithm}
\hfill
In questo caso, il \textit{while} ci porterebbe a calcolare troppe volte le distanze, anche quando non necessario. Infatti, se dopo un nodo \textit{x}, occorre calcolare la distanza del suo figlio \textit{x+1}, sicuramente questo avrà distanza $Dist(x)+1$, senza dover ricalcolarla da zero. \\
Per ovviare a questo problema e raggiungere un costo di $O(n)$:
\begin{algorithm}
	\caption{Esercizio 5}\label{alg:es5-2}
	\begin{algorithmic}[1]
		\Function{contaDistanze}{}
		\State $i \gets -1$
		\For { $i \gets 1$ to \textit{n}}
			\If { $Dist(i) = -1$ }
				\State $calcola(i)$
			\EndIf
		\EndFor
		\EndFunction
	\end{algorithmic}
	\begin{algorithmic}[1]
		\Function{calcola(i)}{}
		\If { $i = u$ }
			\State $ Dist(i) = 0 $
		\ElsIf { $Dist(P[i]) \neq -1$ }
			\State $ Dist(u) = Dist(P[i]) + 1$
		\Else
			\State $int d = calcola(P[i])$
			\State $Dist(i) = d+1$
		\EndIf
		\EndFunction
	\end{algorithmic}
\end{algorithm}
\newpage

\paragraph{6}
Ho un grafo connesso, i cui archi sono colorati. Voglio dimostrare la validità di un algoritmo che produca un \textit{albero di copertura} che utilizza il minor numero di colori distinti. Algoritmo proposto: \hfill
\begin{algorithm}
	\caption{Esercizio 6}\label{alg:es6}
	\begin{algorithmic}[1]
		\Function{algo}{}
		\State $Sol \gets \phi$
		\While {$ E \neq \phi $}
		\If { esiste un arco in \textit{E} con un colore già inserito in \textit{Sol} }
		\State estrai quell'arco
		\Else
		\State estrai arco \textit{e} che ha colore che occorre più spesso in \textit{E}
		\EndIf
		\If { \textit{e} non crea ciclo in \textit{Sol} }
		\State $Sol \gets Sol \cup \{e\}$
		\EndIf
		\EndWhile
		\State \textbf{return} \textit{Sol}
		\EndFunction
	\end{algorithmic}
\end{algorithm} \\ \hfill
\textbf{Soluzione:} Questo algoritmo fa leva sui colori che compaiono maggiormente nel grafo. Il problema è che esistono altri criteri che valorizzano dei colori piuttosto che altri, che vanno a scontrarsi con la regola del \textit{colore che compare più volte}: per esempio, che un dato arco di un dato colore è l'unico che rende il grafo connesso.