\section{Note}
Argomento del corso: preparare gli studenti alla scrittura di algoritmi \textit{efficienti}.
Un algoritmo è definito \textit{inefficiente} se la sua complessità è esponenziale.\\
La maggior parte dei problemi non artefatti sono risolvibili in tempi esponenziali o polinomiali.

\section{Grafi}
I grafi sono \textit{modelli} utilizzati per semplificare lo studio di problemi e la ricerca di soluzioni ad essi.
\subsection{Grafi e Alberi}
I grafi sono considerabili come una "sovraclasse" degli \textit{alberi}.\\
Le differenze principali sono:
\begin{itemize}
	\item i grafi non sono necessariamente connessi (non tutti i nodi sono collegati);
	\item i grafi possono avere cicli.
	
\end{itemize}
Appare chiaro che utilizzare grafi diversi dagli alberi è molto utile se si necessita di rappresentare strutture non gerarchiche.\\
Notare inoltre che, al contrario dei grafi, in un albero il numero complessivo di archi in un albero è necessariamente $ n-1 $.


\subsection{Composizione}
Sono composti da due entità:
\begin{itemize}
	\item \textbf{nodi}: anche detti \textit{vertici}; il loro insieme V è rappresentato come insieme dei loro nomi. Generalmente, la cardinalità di V è indicata con \textit{n}.
	\item \textbf{archi}: collegano i vari nodi. L'insieme E degli archi di un grafo è rappresentato dai nomi delle coppie di nodi che essi collegano. Possono essere unidirezionali o bidirezionali.
\end{itemize}
Si distinguono in \textit{grafi diretti} (con archi unidirezionali) e \textit{grafi indiretti} (con archi bidirezionali).
\paragraph{Un pozzo.}
In un grafo diretto, si definisce \textit{pozzo} un nodo da cui non esce nessun arco.
\paragraph{Un pozzo universale.}
In un grafo diretto, si definisce \textit{pozzo universale} un pozzo verso cui,direttamente o indirettamente, puntano tutti gli altri nodi.

\paragraph{Un esercizio intrigante.}
Dato un grafo, trovare l'algoritmo che restituisce in $ O(n^2) $ \textit{vero} solo se è presente in esso un pozzo universale.

\subsection{Grafi diretti}
Sono bellissimi.
\subsection{Rappresentazione tramite matrice}
\subsection{Rappresentazione tramite lista di adiacenze}