\paragraph{1}
Dato un grfo connesso \textit{G}, dimostrare che - essendo connesso - ci sono almeno due nodi con lo stesso grado: $ \exists u,v \in V : deg(u) = deg(v) $ \\
\textbf{Dimostrazione}: per assurdo consideriamo un grafo connesso che però abbia tutti i nodi di grado diverso. Ci troveremo necessariamente ad avere un nodo con grado $n-1$, un altro con grado $n-2$, fino ad uno con grado $n-n = 0$. Un grado però non può avere grado \textit{0}, a meno che non si tratti di un grafo sconnesso.

\paragraph{2}
Dato un grafo \textit{G}, con \textit{V} nodi e \textit{E} vertici, in cui ciascun nodo abbia grado $\geq 2$, dimostrare che il grafo \textit{G} è ciclico. \\
\textbf{Dimostrazione}: ???

\paragraph{3}
Dato un albero \textit{T} con $n \geq 2$ nodi, dimostrare che esistono almeno 2 foglie. \\
\textbf{Dimostrazione}: visitiamo il grafo partendo da un nodo arbitrario tramite la procedura \textit{DFS}. Questa troverà necessariamente un nodo il cui grado è più alto di tutti gli altri e fermerà la ricorsione. Si tratta della prima foglia. \\
Applicando la stessa procedura, a partire dalla foglia appena individuata, sicuramente, ancora una volta, l'algoritmo, applicherà la ricorsione fino a trovare un nodo il cui grado sia diverso da quello della foglia usata in questo caso come radice. Si tratta della seconda foglia.

\paragraph{4}
Dato un grafo diretto \textit{G}, si vuole calcolare il tempo per calcolarne il grado uscente e normale di ciascun vertice, usando liste di adiacenza. Successivamente, utilizzando matrici. \\
\textbf{Soluzione}: ???